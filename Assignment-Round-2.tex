% Options for packages loaded elsewhere
\PassOptionsToPackage{unicode}{hyperref}
\PassOptionsToPackage{hyphens}{url}
%
\documentclass[
]{article}
\usepackage{lmodern}
\usepackage{amsmath}
\usepackage{ifxetex,ifluatex}
\ifnum 0\ifxetex 1\fi\ifluatex 1\fi=0 % if pdftex
  \usepackage[T1]{fontenc}
  \usepackage[utf8]{inputenc}
  \usepackage{textcomp} % provide euro and other symbols
  \usepackage{amssymb}
\else % if luatex or xetex
  \usepackage{unicode-math}
  \defaultfontfeatures{Scale=MatchLowercase}
  \defaultfontfeatures[\rmfamily]{Ligatures=TeX,Scale=1}
\fi
% Use upquote if available, for straight quotes in verbatim environments
\IfFileExists{upquote.sty}{\usepackage{upquote}}{}
\IfFileExists{microtype.sty}{% use microtype if available
  \usepackage[]{microtype}
  \UseMicrotypeSet[protrusion]{basicmath} % disable protrusion for tt fonts
}{}
\makeatletter
\@ifundefined{KOMAClassName}{% if non-KOMA class
  \IfFileExists{parskip.sty}{%
    \usepackage{parskip}
  }{% else
    \setlength{\parindent}{0pt}
    \setlength{\parskip}{6pt plus 2pt minus 1pt}}
}{% if KOMA class
  \KOMAoptions{parskip=half}}
\makeatother
\usepackage{xcolor}
\IfFileExists{xurl.sty}{\usepackage{xurl}}{} % add URL line breaks if available
\IfFileExists{bookmark.sty}{\usepackage{bookmark}}{\usepackage{hyperref}}
\hypersetup{
  hidelinks,
  pdfcreator={LaTeX via pandoc}}
\urlstyle{same} % disable monospaced font for URLs
\usepackage[margin=1in]{geometry}
\usepackage{color}
\usepackage{fancyvrb}
\newcommand{\VerbBar}{|}
\newcommand{\VERB}{\Verb[commandchars=\\\{\}]}
\DefineVerbatimEnvironment{Highlighting}{Verbatim}{commandchars=\\\{\}}
% Add ',fontsize=\small' for more characters per line
\usepackage{framed}
\definecolor{shadecolor}{RGB}{248,248,248}
\newenvironment{Shaded}{\begin{snugshade}}{\end{snugshade}}
\newcommand{\AlertTok}[1]{\textcolor[rgb]{0.94,0.16,0.16}{#1}}
\newcommand{\AnnotationTok}[1]{\textcolor[rgb]{0.56,0.35,0.01}{\textbf{\textit{#1}}}}
\newcommand{\AttributeTok}[1]{\textcolor[rgb]{0.77,0.63,0.00}{#1}}
\newcommand{\BaseNTok}[1]{\textcolor[rgb]{0.00,0.00,0.81}{#1}}
\newcommand{\BuiltInTok}[1]{#1}
\newcommand{\CharTok}[1]{\textcolor[rgb]{0.31,0.60,0.02}{#1}}
\newcommand{\CommentTok}[1]{\textcolor[rgb]{0.56,0.35,0.01}{\textit{#1}}}
\newcommand{\CommentVarTok}[1]{\textcolor[rgb]{0.56,0.35,0.01}{\textbf{\textit{#1}}}}
\newcommand{\ConstantTok}[1]{\textcolor[rgb]{0.00,0.00,0.00}{#1}}
\newcommand{\ControlFlowTok}[1]{\textcolor[rgb]{0.13,0.29,0.53}{\textbf{#1}}}
\newcommand{\DataTypeTok}[1]{\textcolor[rgb]{0.13,0.29,0.53}{#1}}
\newcommand{\DecValTok}[1]{\textcolor[rgb]{0.00,0.00,0.81}{#1}}
\newcommand{\DocumentationTok}[1]{\textcolor[rgb]{0.56,0.35,0.01}{\textbf{\textit{#1}}}}
\newcommand{\ErrorTok}[1]{\textcolor[rgb]{0.64,0.00,0.00}{\textbf{#1}}}
\newcommand{\ExtensionTok}[1]{#1}
\newcommand{\FloatTok}[1]{\textcolor[rgb]{0.00,0.00,0.81}{#1}}
\newcommand{\FunctionTok}[1]{\textcolor[rgb]{0.00,0.00,0.00}{#1}}
\newcommand{\ImportTok}[1]{#1}
\newcommand{\InformationTok}[1]{\textcolor[rgb]{0.56,0.35,0.01}{\textbf{\textit{#1}}}}
\newcommand{\KeywordTok}[1]{\textcolor[rgb]{0.13,0.29,0.53}{\textbf{#1}}}
\newcommand{\NormalTok}[1]{#1}
\newcommand{\OperatorTok}[1]{\textcolor[rgb]{0.81,0.36,0.00}{\textbf{#1}}}
\newcommand{\OtherTok}[1]{\textcolor[rgb]{0.56,0.35,0.01}{#1}}
\newcommand{\PreprocessorTok}[1]{\textcolor[rgb]{0.56,0.35,0.01}{\textit{#1}}}
\newcommand{\RegionMarkerTok}[1]{#1}
\newcommand{\SpecialCharTok}[1]{\textcolor[rgb]{0.00,0.00,0.00}{#1}}
\newcommand{\SpecialStringTok}[1]{\textcolor[rgb]{0.31,0.60,0.02}{#1}}
\newcommand{\StringTok}[1]{\textcolor[rgb]{0.31,0.60,0.02}{#1}}
\newcommand{\VariableTok}[1]{\textcolor[rgb]{0.00,0.00,0.00}{#1}}
\newcommand{\VerbatimStringTok}[1]{\textcolor[rgb]{0.31,0.60,0.02}{#1}}
\newcommand{\WarningTok}[1]{\textcolor[rgb]{0.56,0.35,0.01}{\textbf{\textit{#1}}}}
\usepackage{graphicx}
\makeatletter
\def\maxwidth{\ifdim\Gin@nat@width>\linewidth\linewidth\else\Gin@nat@width\fi}
\def\maxheight{\ifdim\Gin@nat@height>\textheight\textheight\else\Gin@nat@height\fi}
\makeatother
% Scale images if necessary, so that they will not overflow the page
% margins by default, and it is still possible to overwrite the defaults
% using explicit options in \includegraphics[width, height, ...]{}
\setkeys{Gin}{width=\maxwidth,height=\maxheight,keepaspectratio}
% Set default figure placement to htbp
\makeatletter
\def\fps@figure{htbp}
\makeatother
\setlength{\emergencystretch}{3em} % prevent overfull lines
\providecommand{\tightlist}{%
  \setlength{\itemsep}{0pt}\setlength{\parskip}{0pt}}
\setcounter{secnumdepth}{-\maxdimen} % remove section numbering
\ifluatex
  \usepackage{selnolig}  % disable illegal ligatures
\fi

\author{}
\date{\vspace{-2.5em}}

\begin{document}

Parik -

I'd like you to take the German Credit Data example that you shared, and
write two pieces of R code: one that will train a model (of your
choosing) and save the trained model,

and another that has a prediction function which takes a record (or a
test csv file) as input, loads the trained model, and outputs
corresponding predictions.

What I'm looking for is best coding practices (appropriate commenting,
variable naming conventions, etc.).

Instead of emailing me those directly, I'd like you to upload them to a
Github repo that I can access. You are encouraged to add any explanation
in a README.md.

Load the dataset first.

Downloaded from:
\url{https://online.stat.psu.edu/stat508/resource/analysis/gcd}

We give it the generic name ``data'', as the originally loaded data set.

\begin{Shaded}
\begin{Highlighting}[]
\NormalTok{data }\OtherTok{=} \FunctionTok{read.csv}\NormalTok{(}\StringTok{"german\_credit.csv"}\NormalTok{, }
                \AttributeTok{header =} \ConstantTok{TRUE}\NormalTok{, }
                \AttributeTok{sep =} \StringTok{","}\NormalTok{)}
\end{Highlighting}
\end{Shaded}

Get the names of the variables in the set:

\begin{Shaded}
\begin{Highlighting}[]
\FunctionTok{names}\NormalTok{(data)}
\end{Highlighting}
\end{Shaded}

\begin{verbatim}
##  [1] "Creditability"                     "Account.Balance"                  
##  [3] "Duration.of.Credit..month."        "Payment.Status.of.Previous.Credit"
##  [5] "Purpose"                           "Credit.Amount"                    
##  [7] "Value.Savings.Stocks"              "Length.of.current.employment"     
##  [9] "Instalment.per.cent"               "Sex...Marital.Status"             
## [11] "Guarantors"                        "Duration.in.Current.address"      
## [13] "Most.valuable.available.asset"     "Age..years."                      
## [15] "Concurrent.Credits"                "Type.of.apartment"                
## [17] "No.of.Credits.at.this.Bank"        "Occupation"                       
## [19] "No.of.dependents"                  "Telephone"                        
## [21] "Foreign.Worker"
\end{verbatim}

So, perform analysis on this dataset we must use logistic regression.

We also have two separate CSV files given by the PSU website where we
got the original dataset from. One is a ``Train'' set and the other is a
``Test'' set. Both hail from the same original set, but they were
preprocessed by the same host.

Load them both:

\begin{Shaded}
\begin{Highlighting}[]
\NormalTok{train\_data }\OtherTok{=} \FunctionTok{read.csv}\NormalTok{(}\StringTok{"Training50.csv"}\NormalTok{, }
                \AttributeTok{header =} \ConstantTok{TRUE}\NormalTok{, }
                \AttributeTok{sep =} \StringTok{","}\NormalTok{)}

\NormalTok{test\_data }\OtherTok{=} \FunctionTok{read.csv}\NormalTok{(}\StringTok{"Test50.csv"}\NormalTok{, }
                \AttributeTok{header =} \ConstantTok{TRUE}\NormalTok{, }
                \AttributeTok{sep =} \StringTok{","}\NormalTok{)}
\end{Highlighting}
\end{Shaded}

We want the response variable, or the outcome variable to be
``Creditability''.

This is a binary variable that decides whether or not the loan will be
defaulted.

It can be influenced by a variety of factors - to name a few, they could
include the applicant's duration of credit, credit amount, number of
dependents and age.

Here, we see from examining the CSV file, that the variables that have
the most obvious meaning are:

Duration of credit Credit Amount Length of current employment
Installment percent Age Number of dependents

By looking at the values, we see that they are ordinal as well as
continuous, not categorical, meaning that as we only want to pick
variables that we expect to obviously explain whether or not an
applicant will have their loan rejected.

First we want to check which predictors are significant in influencing
the response, or the chance of defaulting.

We will perform a logistic regression model on the response vs.~the
predictor variables selected, as mentioned above.

If the p-values of each of the estimated coefficients of the predictor
variables are less than a recommended cutoff of 0.05, we will deem them
as insignificant.

Create the model for the full dataset:

\begin{Shaded}
\begin{Highlighting}[]
\CommentTok{\#Full\_model performs logistic regression on the full dataset, not on every variable in the set!}
\NormalTok{full\_model }\OtherTok{=} \FunctionTok{glm}\NormalTok{(}
\NormalTok{                  Creditability }\SpecialCharTok{\textasciitilde{}}\NormalTok{ Duration.of.Credit..month.}
                            \SpecialCharTok{+}\NormalTok{ Credit.Amount}
                            \SpecialCharTok{+}\NormalTok{ Length.of.current.employment}
                            \SpecialCharTok{+}\NormalTok{ Instalment.per.cent}
                            \SpecialCharTok{+}\NormalTok{ Age..years.}
                            \SpecialCharTok{+}\NormalTok{ No.of.dependents, }
                  \AttributeTok{data =}\NormalTok{ train\_data, }
                  \AttributeTok{family =}\NormalTok{ binomial}
\NormalTok{                 )}
\end{Highlighting}
\end{Shaded}

Now check the p-values using the summary, if they are less than 0.05:

\begin{Shaded}
\begin{Highlighting}[]
\FunctionTok{which}\NormalTok{(}\FunctionTok{summary}\NormalTok{(full\_model)}\SpecialCharTok{$}\NormalTok{coefficients[,}\DecValTok{4}\NormalTok{] }\SpecialCharTok{\textless{}} \FloatTok{0.05}\NormalTok{)}
\end{Highlighting}
\end{Shaded}

\begin{verbatim}
##                  (Intercept)   Duration.of.Credit..month. 
##                            1                            2 
##                Credit.Amount Length.of.current.employment 
##                            3                            4 
##          Instalment.per.cent 
##                            5
\end{verbatim}

It turns out that the duration of credit month, credit amount, length of
current employment and installment percent are significant. Another two
variables were not. So we include those in our training model.

First train the model.

\begin{Shaded}
\begin{Highlighting}[]
\NormalTok{train\_model }\OtherTok{=} \FunctionTok{glm}\NormalTok{(}
\NormalTok{                  Creditability }\SpecialCharTok{\textasciitilde{}}\NormalTok{ Duration.of.Credit..month.}
                                \SpecialCharTok{+}\NormalTok{ Credit.Amount}
                                \SpecialCharTok{+}\NormalTok{ Length.of.current.employment}
                                \SpecialCharTok{+}\NormalTok{ Instalment.per.cent}
                                \SpecialCharTok{+}\NormalTok{ Age..years.}
                                \SpecialCharTok{+}\NormalTok{ No.of.dependents, }
                  \AttributeTok{data =}\NormalTok{ train\_data, }
                  \AttributeTok{family =}\NormalTok{ binomial}
\NormalTok{                 )}
\end{Highlighting}
\end{Shaded}

We want to check the accuracy or performance of the trained model when
predicting train values and test values. Since the response is binary
and we used logistic regression, we have to compare the binary values of
our predicted responses to the actual response values from the
respective data sets.

But the output predicted values are not in terms of 0 or 1 - they are
continuous. We must classify them by using the standard cutoff as 0.5
for if the value is less than 0.5 and 1 if it is more than 0.5.

Now we will predict Creditability values using train data, or the train
dataset we loaded earlier, to predict train response values based on the
train model.

We will call the predicted response values ``y\_train''.

\begin{Shaded}
\begin{Highlighting}[]
\CommentTok{\#Predict the values using train model on train data:}
\NormalTok{y }\OtherTok{=} \FunctionTok{predict}\NormalTok{(train\_model, train\_data, }\AttributeTok{type =} \StringTok{"response"}\NormalTok{)}

\CommentTok{\#Convert/classify those responses in terms of 0 and 1.}
\CommentTok{\#These are the corresponding predictions.}
\NormalTok{y\_train }\OtherTok{=} \FunctionTok{ifelse}\NormalTok{(y }\SpecialCharTok{\textgreater{}} \FloatTok{0.5}\NormalTok{, }\DecValTok{1}\NormalTok{, }\DecValTok{0}\NormalTok{)}
\end{Highlighting}
\end{Shaded}

Now we will test the model using test data, or the test dataset we
loaded earlier, to predict test response values based on the train
model.

We will call the predicted response values ``y\_test''.

\begin{Shaded}
\begin{Highlighting}[]
\CommentTok{\#Predict the values using train model on test data:}
\NormalTok{y }\OtherTok{=} \FunctionTok{predict}\NormalTok{(train\_model, test\_data, }\AttributeTok{type =} \StringTok{"response"}\NormalTok{)}

\CommentTok{\#Convert/classify those responses in terms of 0 and 1.}
\CommentTok{\#These are the corresponding predictions.}
\NormalTok{y\_test }\OtherTok{=} \FunctionTok{ifelse}\NormalTok{(y }\SpecialCharTok{\textgreater{}} \FloatTok{0.5}\NormalTok{, }\DecValTok{1}\NormalTok{, }\DecValTok{0}\NormalTok{)}
\end{Highlighting}
\end{Shaded}

There are 500 values for the corresponding predictions for each of the
train and test data. Rather than output every single value, our goal
should be to generate them for both train and test data and check how
those values match with the actual response values from the dataset - in
other words, the performance of the model.

To compare the performance of the train model on the test and train
data, and how equal the train and test response predicted values,

we count how many predicted test/train values out of the total number of
dataset observations

First for the train responses:

\begin{Shaded}
\begin{Highlighting}[]
\FunctionTok{mean}\NormalTok{(y\_train }\SpecialCharTok{==}\NormalTok{ train\_data}\SpecialCharTok{$}\NormalTok{Creditability)}
\end{Highlighting}
\end{Shaded}

\begin{verbatim}
## [1] 0.734
\end{verbatim}

Now for the test responses:

\begin{Shaded}
\begin{Highlighting}[]
\FunctionTok{mean}\NormalTok{(y\_test }\SpecialCharTok{==}\NormalTok{ test\_data}\SpecialCharTok{$}\NormalTok{Creditability)}
\end{Highlighting}
\end{Shaded}

\begin{verbatim}
## [1] 0.682
\end{verbatim}

As we see here, the model performs reasonably well on the train data,
outputting the correct response values more than 70\% of the time.
However, on the test data, it only gets it right a little less than 70\%
of the time. While this is not abnormal, we should recognize that there
are indeed underlying potential improvements that could have been made
when fitting the model, especially when deciding what predictor
variables to use to predict the response. Other than that, this accuracy
rate is reasonably good.

\end{document}
